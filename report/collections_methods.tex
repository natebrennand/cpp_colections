

\subsection{Member Functions}


\subsubsection{\code{Collection<T>()}}

Construct an empty Collection.

\begin{lstlisting}[title=example]
auto empty_collection = Collection<int>();
\end{lstlisting}




\subsubsection{\code{Collection<T>(int size)}}

Construct a pre-sized Collection.

\begin{lstlisting}[title=example]
auto presized_collection = Collection<int>(5);
\end{lstlisting}




\subsubsection{\code{Collection<T>(std::vector<T> list)}}

Construct a Collection from a \code{std::vector}.

\begin{lstlisting}[title=example]
auto vector_collection = Collection<int>(std::vector<int> {1, 2, 3});
\end{lstlisting}




\subsubsection{\code{Collection<T>(std::list<T> list)}}

Construct a Collection from a \code{std::list}.

\begin{lstlisting}[title=example]
auto list_collection = Collection<int>(std::list<int> {1, 2, 3});
\end{lstlisting}





\subsubsection{\code{Collection<T>(std::array<T, size> list)}}

Construct a Collection from a \code{std::array}.

\begin{lstlisting}[title=example]
auto array_collection = Collection<int>(std::array<int,3> {1, 2, 3});
\end{lstlisting}




\subsubsection{\code{Collection<T>(T d[], int len)}}

Construct a Collection from a C-style array (requires length).

\begin{lstlisting}[title=example]
int int_c_array[] {1, 2, 3};
auto c_array_collection = Collection<int>(int_c_array, 3);
\end{lstlisting}




\subsubsection{\code{Collection<T>::vector()}}

Returns the Collection as a \code{std::vector}.




\subsubsection{\code{Collection<T>::list()}}

Returns the Collection as a \code{std::list}.




\subsubsection{\code{Collection<T>::size()}}

Returns an \code{int} with the current size of the Collection.




\subsubsection{\code{Collection<T>::head()}}

Returns the first element in the Collection.

\begin{lstlisting}[title=example]
auto a = range(1, 10);
std::cout << a.head(); << std::endl;

// 1
\end{lstlisting}




\subsubsection{\code{Collection<T>::last()}}

Returns the last element in the Collection.

\begin{lstlisting}[title=example]
auto a = range(1, 10);
std::cout << a.last(); << std::endl;

// 9
\end{lstlisting}




\subsubsection{\code{Collection<T>::init()}}

Return all elements except the last.

\begin{lstlisting}[title=example]
auto a = range(5);
std::cout << a.init() << std::endl;

// [0,1,2,3]
\end{lstlisting}




\subsubsection{\code{Collection<T>::tail()}}

Return all elements except the head.

\begin{lstlisting}[title=example]
auto a = range(5);
std::cout << a.tail() << std::endl;

// [1,2,3,4]
\end{lstlisting}




\subsubsection{\code{Collection<T>::pop\_head()}}

Remove the head of the collection.

\begin{lstlisting}[title=example]
auto a = range(5);
a.pop_head();
std::cout << a << std::endl;

// [1,2,3,4]
\end{lstlisting}




\subsubsection{\code{Collection<T>::each(std::function<void(T)> func)}}

Apply a function to all the elements in the Collection

\begin{lstlisting}[title=example]
int sum = 0;
auto a = range(5);
a.each([&](int x) {
    sum += x;
});
std::cout << sum << std::endl;

// 10
\end{lstlisting}




\subsubsection{\code{Collection<T>::filter(std::function<bool(T)> func)}}

Return a subset of the Collection containing the elements of the original Collection that pass a predicate function.

\begin{lstlisting}[title=example]
auto a = range(1,11);
std::cout << a.filter([](int x) { return x % 2 == 0; }) << std::endl;

// [2,4,6,8,10]
\end{lstlisting}




\subsubsection{\code{Collection<T>::slice(int low, int high)}}

Return the elements whose indices are within the range \code{[low, high)}.

\begin{lstlisting}[title=example]
auto a = range(10, 20);
std::cout << a.slice(1,4) << std::endl;

// [11,12,13]
\end{lstlisting}





\subsubsection{\code{Collection<T>::map(Function func)}}

Return the Collection that results from the transformation of each element in the original Collection.

\begin{lstlisting}[title=example]
auto a = range(3);
std::cout << a.map([](int x) { return x+1; }) << std::endl;

// [1,2,3]
\end{lstlisting}




\subsubsection{\code{Collection<T>::tmap(Function func, int threads)}}

An alternative implementation of map that uses multiple concurrent
std::threads to speed up processing.

\begin{lstlisting}[title=example]
auto a = range(3);
std::cout << a.tmap([](int x) { return x+1; }, 3) << std::endl;

// [1,2,3]
\end{lstlisting}





\subsubsection{\code{Collection<T>::reduceLeft(std::function<T(T, T)> func)}}

Return the result of the application of the same binary operator on adjacent pairs of elements in the Collection, starting from the left.

\begin{lstlisting}[title=example]
int sum = range(5).reduceLeft([](int x, int y) { return x+y; });
std::cout << sum << std::endl;

// 10
\end{lstlisting}




\subsubsection{\code{Collection<T>::reduceRight(std::function<T(T, T)> func)}}

Return the result of the application of the same binary operator on adjacent pairs of elements in the Collection, starting from the right.





\subsubsection{\code{Collection<T>::treduce(std::function<T(T, T)> func, int threads)}}

An alternative implementation of reduce that uses multiple concurrent threads to speed up processing
(note that the function passed to treduce must be commutative to achieve accurate result).

\begin{lstlisting}[title=example]
int sum = range(5).treduce([](int x, int y) { return x+y; }, 3);
std::cout << sum << std::endl;

// 10
\end{lstlisting}




\subsubsection{\code{Collection<T>::foldLeft(Function func, U init)}}

Return the result of the application of the same binary operator on all elements in the Collection as well as an initial value, starting from the left.

\begin{lstlisting}[title=example]
int sum = range(5).foldLeft([](int x, int y) { return x+y; }, 0);
std::cout << sum << std::endl;

// 10
\end{lstlisting}




\subsubsection{\code{Collection<T>::foldRight(Function func, U init)}}

Return the result of the application of the same binary operator on all elements in the Collection as well as an initial value, starting from the right.




\subsubsection{\code{Collection<T>::scanLeft(Function func, U init)}}

Returns the intermediate results of the binary accumulation of the elements in a Collection as well as an initial value, starting from the left.

\begin{lstlisting}[title=example]
auto ints = range(1,5);
auto ints2 = ints.scanLeft([](int x, int y) { return x+y; }, 1);
std::cout << ints2 << std::endl;

// [1,2,4,7,11]
\end{lstlisting}




\subsubsection{\code{Collection<T>::scanRight(Function func, U init)}}

Returns the intermediate results of the binary accumulation of the elements in a Collection as well as an initial value, starting from the left.

\begin{lstlisting}[title=example]
auto ints = range(1,5);
auto ints2 = ints.scanRight([](int x, int y) { return x+y; }, 1);
std::cout << ints2 << std::endl;

// [11,10,8,5,1]
\end{lstlisting}




\subsection{Non-member Functions}

\subsubsection{\code{concat()}}

Return the Collection that results from the concatentation of an arbitrary number of Collections of the same type.

\begin{lstlisting}[title=example]
auto a = range(3);
auto b = range(3);
auto c = range(3);
std::cout << concat(a, b, c) << std::endl;

// [0,1,2,0,1,2,0,1,2]
\end{lstlisting}





\subsubsection{\code{range(T size)}}

Return Collection of numeric types over the range \code{[0, size)}.

\begin{lstlisting}[title=example]
auto a = range(5);
std::cout << a << std::endl;

// [0,1,2,3,4]
\end{lstlisting}




\subsubsection{range(T low, T high)}

Return Collection of numeric types over the range \code{[low, high)}.

\begin{lstlisting}[title=example]
auto a = range(5, 10);
std::cout << a << std::endl;

// [5,6,7,8,9]
\end{lstlisting}





\subsubsection{zip(Collection<U>... other)}

Return a Collection of tuples, where each tuple contains the elements of the zipped Collections that occur at the same position.

\begin{lstlisting}[title=example]
auto a = range(3);
auto b = range(3.0);
auto c = Collection(std::vector<char> {'a','b','c'});

auto d = zip(a, b, c);

assert(d[0] == std::make_tuple(0, 0.0, 'a'));
\end{lstlisting}




\subsubsection{zipWith(Function func, Collection<U>... other)}
Generalizes \code{zip} by zipping with the function given as the first argument instead of a tupling function.

\begin{lstlisting}[title=example]
auto a = range(3);
auto b = range(3);
std::cout << zipWith([](int x, int y) { return x+y; }, a, b) << std::endl;

// [0,2,4]
\end{lstlisting}


