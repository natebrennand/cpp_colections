
\subsection{Member Functions}

\subsubsection{\code{Stream<T>(T head, std::function<T>()> gen)}}

Default Stream constructor.
Takes an initial head value, and a function that returns a Stream.
The programmer should not have to interact with this method directly.
Instead, use the \code{def\_generator} macro to define a Stream generator.


\begin{lstlisting}[title=example]
std::function<Stream<int>()> ones = [&]() { return Stream<int>(1, ones) };
std::cout << ones().take(3) << std::endl;

# [1,1,1]
\end{lstlisting}

\subsubsection{\code{Stream<T>(T head, Stream<T> tail)}}

Tail-only Stream constructor.
The programmer should not have to interact with this method directly.

\subsubsection{\code{Stream<T>::head()}}

Return the first element of the Stream.

\begin{lstlisting}[title=example]
std::cout << from(1).head() << std::endl;

# 1
\end{lstlisting}

\subsubsection{\code{Stream<T>::tail()}}

Return the Stream minus the current head.

\begin{lstlisting}[title=example]
std::cout << from(1).tail().head() << std::endl;

# 2
\end{lstlisting}

\subsubsection{\code{Stream<T>::take(int n)}}

Return \code{n} elements taken from the Stream as a Collection.

\begin{lstlisting}[title=example]
std::cout << from(1).take(5) << std::endl;

# [1,2,3,4,5]
\end{lstlisting}

\subsubsection{\code{Stream<T>::filter(std::function<bool(T)> func)}}

Return a Stream that contains the elements of the initial Stream that match the predicate function.

\begin{lstlisting}[title=example]
auto evens = from(1).filter([](int x) { return x % 2 == 0; });
std::cout << evens.take(5) << std::endl;

# [2,4,6,8,10]
\end{lstlisting}

\subsubsection{\code{Stream<T>::map(Function func)}}

Return the Stream that results from the transformation of each element in the original Stream.

\begin{lstlisting}[title=example]
auto squares = from(1).map([](int x) { return x * x; });
std::cout << squares.take(5) << std::endl;

# [1,4,9,16,25]
\end{lstlisting}

\subsection{Non-member Functions}

\subsubsection{\code{cons(T value, Stream<T> other)}}

Prepends a value to a Stream.

\begin{lstlisting}[title=example]
auto ints = from(1);
std:cout << cons(10, ints).take(5) << std::endl;

# [10,1,2,3,4]
\end{lstlisting}

\subsubsection{\code{operator\&(T value, Stream<T> other)}}

Prepends a value to a Stream.

\begin{lstlisting}[title=example]
auto ints = from(1);
std:cout << (10 & (20 & ints)).take(5) << std::endl;

# [10,20,1,2,3]
\end{lstlisting}

\subsubsection{\code{from(T n, T step=1)}}

Construct a Stream, starting at n, incrementing by step (defaults to 1).

\begin{lstlisting}[title=example]
std:cout << from(1).take(5) << std::endl;

# [1,2,3,4,5]
\end{lstlisting}

\subsubsection{\code{zip(Stream<U>... other)}}

Return a Stream of tuples, where each tuple contains the elements of the zipped Streams that occur at the same position

\subsubsection{\code{zipWith(Function func, Stream<U>... other)}}

Generalizes zip by zipping with the function given as the first argument instead of a tupling function.

\subsection{Macros}

\subsubsection{\code{def\_generator(name, return\_type, arg\_types...)}}

A macro to ease the syntax of defining any arbitrary recursive Stream generator.

\begin{lstlisting}[title=example]
def_generator(ones, int) {
    return Stream<int>(1, ones)
};
std::cout << ones().take(5) << std::endl;

# [1,1,1,1,1]
\end{lstlisting}

\begin{lstlisting}[title=example]
def_generator(fibs, int, int prev, int curr) {
    return Stream<int>(curr, [=]() { return fibs(curr, prev+curr); });
};
std::cout << fibs(0, 1).take(10) << std::endl;

# [1,1,2,3,5,8,13,21,34,55]
\end{lstlisting}

